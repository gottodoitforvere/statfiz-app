\documentclass[12pt,a4paper]{article}
\usepackage[utf8]{inputenc}
\usepackage[russian]{babel}
\usepackage{amsmath, amssymb}
\usepackage{geometry}
\usepackage{graphicx}
\usepackage{hyperref}
\geometry{margin=2.2cm}

\title{Градиент температур и термодинамические основы симуляции}
\author{Учебная демонстрация по статистической физике}
\date{}

\begin{document}

\maketitle

\tableofcontents
\newpage

\section{Первый закон термодинамики}
Первый закон термодинамики описывает энергетический баланс между системой и её окружением:
\begin{equation}
    \mathrm{d}Q = \mathrm{d}A + \mathrm{d}U,
\end{equation}
где $\mathrm{d}Q$~--- элементарное количество теплоты, $\mathrm{d}A$~--- работа внешних сил, а $\mathrm{d}U$~--- изменение внутренней энергии системы. Записывая среднее значение гамильтониана $H(z; a)$ по распределению $w(z; a)$, получаем внутреннюю энергию
\begin{equation}
    U(a) = \langle H \rangle = \int H(z; a) w(z; a)\, \mathrm{d}z.
\end{equation}

Работа появляется только при изменении внешних параметров $a_i$:
\begin{equation}
    \mathrm{d}A = \sum_i \langle B_i \rangle\, \mathrm{d}a_i, \qquad B_i = \frac{\partial H}{\partial a_i},
\end{equation}
а передача энергии без изменения $a_i$ интерпретируется как теплообмен
\begin{equation}
    \mathrm{d}Q = \int H(z; a)\, \delta w(z; a)\, \mathrm{d}z.
\end{equation}

В численной модели эти формулы задают эволюцию энергии частиц и взаимодействие с ``тёплыми'' или ``холодными'' стенками. Пересчёт скоростей после столкновений обеспечивает реализацию требований первого закона на каждом шаге интегрирования.

\section{Энтропия и её изменение}
Дифференциал энтропии равновесной системы определяется соотношением Клаузиуса:
\begin{equation}
    \mathrm{d}S = \frac{\mathrm{d}Q}{T}.
\end{equation}
Энтропия является функцией состояния, что позволяет использовать статистическое определение
\begin{equation}
    S = -k_{\mathrm{B}} \int w(z; a) \ln w(z; a)\, \mathrm{d}z + S_0.
\end{equation}

Второй закон термодинамики утверждает, что
\begin{equation}
    \mathrm{d}S \ge \frac{\mathrm{d}Q}{T},
\end{equation}
и равенство достигается только в квазистатических процессах. В симуляции это отражается в различии между медленным и быстрым изменением условий на границе. Например, резкое сжатие приводит к более высокому нагреву газа и большему увеличению энтропии, чем медленное.

\subsection{Неравновесные процессы и метастабильность}
Рассмотрим два теплоизолированных сосуда с поршнями. В одном поршень опускается медленно (почти квазистатически), в другом~--- быстро. После установления равновесия поршень, опущенный резко, остановится выше: газ там получил больше кинетической энергии. На микроскопическом уровне поршень ``сметает'' молекулы при движении вниз, ускоряя их сильнее, чем замедляется при обратном ходе, когда сталкивается с меньшим числом частиц.

В живых тканях и сложных материалах нередко встречаются метастабильные состояния~--- локальные минимумы свободной энергии. Быстрое охлаждение ``замораживает'' их, почти не изменяя энтропию, тогда как медленное охлаждение позволяет системе перейти в более упорядоченное стабильное состояние с меньшей энтропией. Этот эффект важен при хранении биологических образцов и формировании структур в материалах.

\section{Градиент температур и диффузия}
В зазоре между стенками симуляции устанавливается градиент температур: одна стенка поддерживается тёплой, другая~--- холодной. Статистическое описание переноса частиц в такой среде удобно вести с помощью стохастических дифференциальных уравнений (СДУ) и уравнения Фоккера--Планка (УФП).

\subsection{Связь СДУ и УФП}
Одномерное СДУ общего вида
\begin{equation}
    \mathrm{d}x = a(x)\, \mathrm{d}t + b(x)\, \circ \mathrm{d}W_t
\end{equation}
соответствует УФП для плотности вероятности $w(x, t)$:
\begin{equation}
    \frac{\partial w}{\partial t} = -\frac{\partial}{\partial x} \left[ K_1(x) w \right] + \frac{1}{2} \frac{\partial^2}{\partial x^2} \left[ K_2(x) w \right],
\end{equation}
где коэффициенты дрейфа и диффузии в интерпретации Стратоновича выражаются через $a(x)$ и $b(x)$:
\begin{equation}
    K_1(x) = -a(x) + D\, b(x) b'(x), \qquad K_2(x) = 2 D\, b(x)^2.
\end{equation}
Поправка $D\, b(x) b'(x)$ отвечает за направленный перенос в среде с неоднородной ``силой шума''. В интерпретации Ито она отсутствует, что корректно лишь при постоянном $b(x)$.

\subsection{Термодиффузионный дрейф}
Температурный градиент делает движение частиц асимметричным: в более тёплой области средняя скорость молекул больше, и диффузия идёт быстрее. Облако примеси смещается в сторону нагретой части пространства. Этот эффект наблюдается, например, для ароматических молекул: сначала запах ощущается на тёплой стороне.

Стационарное решение УФП при нулевом потоке выражается через коэффициенты СДУ:
\begin{equation}
    w_{\mathrm{st}}(x) = \frac{C}{b(x)} \exp\!\left[ -\frac{1}{D} \int^x \frac{a(x')}{b(x')^2}\, \mathrm{d}x' \right].
\end{equation}
Профиль $w_{\mathrm{st}}(x)$ показывает, как температурный градиент формирует устойчивое распределение плотности газа в симуляции.

\section{Практическое использование в симуляции}
\begin{itemize}
    \item Значения температур стенок определяют аппроксимацию коэффициентов $a(x)$ и $b(x)$, а значит, и стационарное распределение.
    \item При изменении температур важно соблюдать энергетический баланс, чтобы корректно моделировать обмен теплом согласно первому закону.
    \item Контроль метастабильных состояний позволяет исследовать влияние скорости охлаждения на структуру и энтропию системы.
\end{itemize}

\vspace{1em}
\noindent%
Компилируйте документ в PDF (например, \texttt{pdflatex theory\_ru.tex}), сохраните результат под именем \texttt{theory\_ru.pdf} и поместите файл в каталог \texttt{\_internal/theory/} проекта. Интерфейс приложения отобразит страницы автоматически.

\end{document}

